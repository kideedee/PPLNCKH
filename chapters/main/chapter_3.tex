\chapter{Phương pháp giải quyết vấn đề}
\label{chap3}

\section{Hành vi người dùng trên mạng xã hội}

\subsection{Hành vi đánh giá}
Các hệ tư vấn thường được xây dựng từ:
\begin{itemize}
    \item Tập người dùng $W={w_1, w_2, ..., w_n}$
    \item Tập sản phẩm $X={x_1, x_2, ..., x_n}$
\end{itemize}
Hành vi đánh giá là hành động người dùng chấm điểm cho sản phẩm. Thông tin này
được lưu trữ và thường được sử dụng làm cơ sở cho hệ thống thực hiện tư vấn. Điểm
đánh giá từ người dùng $w_i$ cho sản phẩm $x_j$ được định nghĩa như sau:
\begin{equation}
    rating_{ij} = y, ~~~~~~ y \in {1, 2, ..., t}
\end{equation}
Trong đó, $t$ thường được chọn là 5 hoặc 10.

\subsection{Hành vi bình luận}
\label{hvbl}
Hành vi bình luận là hành động của người dùng khi diễn đạt suy nghĩ, quan điểm của
mình bằng văn bản. Người dùng thực hiện hành vi bình luận đối với sản phẩm thay vì
hành vi chấm điểm sẽ mô tả rõ hơn trải nghiệm, suy nghĩ của họ đối với sản phẩm. Mỗi
bình luận $comment_{ij}$ người dùng $w_i$ bày tỏ quan điểm đối với sản phẩm $x_j$. Bình luận
mang nhãn 0 nếu người dùng thích hoặc khen khách sạn. Ngược lại, bình luận mang
nhãn 1 nếu người dùng không thích hoặc chê khách sạn.
\begin{equation}
    comment_{ij} = \left\{
        \begin{array}{ c l }
            0 & \text{nếu $w_i$ thích $x_j$} \\ 
            1 & \text{nếu ngược lại}
        \end{array}
    \right.
\end{equation}

\subsection{Mô hình kết hợp hành vi đánh giá và hành vi bình luận}
Để sử dụng dữ liệu hành vi đánh giá và bình luận cùng lúc cho tư vấn khách sạn thì
cần có một phương pháp để kết hợp hai loại dữ liệu này. Như đã trình bày trong Phần
\ref{hvbl}: $rating_{ij}$: Điểm đánh giá từ người dùng $w_i$ cho sản phẩm $x_j$ và 
$comment_{ij}$: Bình luận bày tỏ quan điểm từ người dùng $w_i$ cho sản phẩm $x_j$.
Theo \cite{yang2022exploring}, các bình luận tiêu cực có ảnh hưởng không nhỏ tới quyết định 
mua hàng của người dùng. Tuy nhiên, nếu sản phẩm có nhiều phản hồi tích cực thì cũng làm tăng 
khả năng mua hàng của người dùng. Do đó, đồ án thực hiện kết hợp dựa trên ý tưởng:
\textbf{“Nếu khách sạn có bình luận tiêu cực thì điểm đánh giá dành cho khách sạn này cần
phải hạ xuống. Tuy nhiên, khách sạn có nhiều phản hồi tích cực thì điểm đánh giá
cũng cần được tăng lên”}. Điều này có nghĩa là, nếu khách sạn có nhiều phản hồi tích
cực thì các điểm đánh giá dành cho khách sạn này sẽ được thưởng thêm và ngược lại,
nếu khách sạn có nhiều bình luận phàn nàn thì điểm đánh giá sẽ bị trừ đi.

Với ý tưởng trên, điểm đánh giá dự đoán $rating_{ij}$ của người dùng $w_i$ với khách sạn
$x_j$ , tỷ lệ số bình luận tích cực $p\_rate_j$, tỷ lệ số bình luận tiêu cực 
$n\_rate_j$ của khách sạn $x_j$ sẽ là 3 thành phần quyết định tới điểm đánh giá cuối 
cùng dành cho  khách sạn. Coi điểm đánh giá cuối cùng là 100\%, $\alpha$ và $\beta$ là 2 trọng 
số tương ứng của $rating_{ij}$, $p\_rate_j$ và $n\_rate_j$ quyết định mức độ ảnh 
hưởng của 2 thành  phần này lên điểm đánh giá cuối cùng. Như vậy, điểm đánh giá cuối cùng 
của người dùng có thể  biểu diễn theo công thức:
\begin{equation}
    c\_rating(w_i, x_j) = \alpha \times rating_{ij} + \beta \times (p\_rate_j - n\_rate_j)
\end{equation}
Trong đó:
\begin{itemize}
    \item $c\_rating(w_i, x_j) \in [0; 5]$ là điểm đánh giá kết hợp, được sử dụng để làm dữ liệu thực
        hiện huấn luyện và đánh giá
    \item $rating_{ij} \in [0; 5]$ là điểm đánh giá được dự đoán thông qua thuật toán lọc 
        cộng tác, 
    \item $p\_rate_j \in [0; 1]$ là tỉ lệ số $comment_{ij}=0$ trong tổng số các 
        $comment_{ij}$ của sản phẩm $x_j$
    \item $n\_rate_j \in [0; 1]$ là tỉ lệ số $comment_{ij}=1$ trong tổng số các 
        $comment_{ij}$ của sản phẩm $x_j$
    \item Do $rating_{ij}$ nằm trong khoảng giá trị khác với $p\_rate_j$ và $n\_rate_j$ 
        vì vậy, trước khi thực hiện kết hợp, đồ án thực hiện chuyển $rating_{ij}$ về cùng 
        khoảng giá trị $[0;1]$ với $p\_rate_j$ và $n\_rate_j$ bằng cách 
        $rating_{ij}=\frac{rating_{ij}}{5}$
    \item Sau khi thực hiện tính toán, để đưa điểm đánh giá dự đoán về khoảng ban đầu,
        ta chỉ cần thực hiện $c\_rating(w_i, x_j) = c\_rating(w_i, x_j) \times 5$
\end{itemize}