% \chapter*{Tóm tắt khóa luận}
\chapter*{\centering\Large{Mở đầu}}
\addcontentsline{toc}{chapter}{Tóm tắt khóa luận}

Cuộc sống của con người ngày càng phát triển, các nhu cầu cá nhân như: giao lưu,
kết bạn, tiêu dùng, du lịch, … ngày tăng. Nhu cầu tiêu dùng ngày càng tăng, cùng với
sự phát triển của công nghệ thống tin, các hệ thống thương mại điện tử ra đời và ngày
càng lớn mạnh, tiêu biểu như: Facebook, Youtube, Tripadvisor, … Những trang thương
mại điện tử này hỗ trợ doanh nghiệp quảng bá sản phẩm tới tay người tiêu dùng nhanh
hơn so với bán hàng truyền thống. Tuy nhiên, khi người dùng được tiếp cận sản phẩm,
dịch vụ một cách nhanh chóng thì họ cũng phải đối mặt với vấn đề có quá nhiều sản
phẩm và dịch vụ và đâu thực sự là sản phẩm họ cần. Đây là tình trạng quá tải thông tin,
khi người dùng có quá nhiều lựa chọn. Tuy nhiên, đôi khi họ cũng phải đối mặt với tình
huống nghịch lý rằng có rất nhiều thông tin, nhưng thường rất khó để có thông tin phù
hợp \cite{edmunds2000problem}. Với hiện trạng nêu trên, nhu cầu cấp thiết đặt ra cần có các hệ thống tự động
hóa, hỗ trợ người dùng lọc thông tin cũng như cá nhân hóa đối với từng người dùng.

Hệ tư vấn ra đời nhằm giải quyết vấn đề quá tải thông tin từ người dùng, giúp họ
khám phá những sản phẩm khác nhau nằm trong sở thích của mình. Có rất nhiều trang
thương mại điện tử lớn sử dụng hệ tư vấn nhằm cải thiện doanh thu và tăng sự thân thiện
với người dùng, một trong số đó là Youtube. Youtube, ra đời vào tháng 2, 2005 với sự
phát triển nhanh chóng đã trở thành nền tảng chia sẻ video trực tuyến lớn nhất hiện nay
với hơn 1 tỷ lượt xem mỗi ngày từ hàng triệu người dùng và mỗi phút có hơn 24 giờ
thời lượng video được tải lên nền tảng này. Hệ tư vấn là một phần trong sự thành công
của Youtube khi đóng góp 60\% lượt bấm xem video từ trang chủ và các video được gợi
ý từ hệ thống có tỷ lệ bấm xem gấp 2 lần những video được nhiều người xem nhất và
được đánh giá cao nhất \cite{davidson2010youtube}.

Một trong các thuật toán tư vấn điển hình và phổ biến là lọc cộng tác và hoạt động
rất hiệu quả. Các hệ tư vấn truyền thống thường sử dụng dữ liệu điểm đánh giá để làm
cơ sở tư vấn. Tuy nhiên, theo \cite{isinkaye2015recommendation}, thuật toán này vẫn còn những vấn đề còn tồn tại như:
\begin{itemize}
    \item Vấn đề người dùng mới, sản phẩm mới (Cold Start)
    \item Vấn đề thưa thớt dữ liệu
\end{itemize}
Do thói quen lười đánh giá từ người dùng, gây ra những vấn đề trên ảnh hưởng tới độ
chính xác của hệ tư vấn lọc cộng tác.

Với sự bùng nổ của các trang thương mại điện tử, các hành vi bày tỏ quan điểm ngày
càng đa dạng và phong phú. Do đó, các phương pháp phân loại văn bản ngày càng được
cải thiện và trở nên chính xác hơn. Những dữ liệu văn bản này cũng mang ý nghĩa bày
tỏ quan điểm đối với sản phẩm.

Để hệ tư vấn có những đề xuất chính xác hơn cũng như tận dụng dữ liệu văn bản
cùng các kỹ thuật phân loại được phát triển, đồ án lựa chọn đề tài \textbf{"Mô hình kết hợp
hành vi đánh giá và bình luận cho tư vấn khách sạn"} với mục tiêu nghiên cứu lý thuyết về hệ tư vấn, các kỹ thuật tư vấn, tiền xử lý văn bản và phân loại văn bản về lĩnh
vực cụ thể là gợi ý các khách sạn trên các bộ dữ liệu thu thập được.

Đồ án được chia thành 4 chương với nội dung như sau:

\textbf{Chương 1: Tổng quan về hệ tư vấn} - Nội dung trong Chương 1 giới thiệu tổng quan về hệ tư vấn và các kỹ thuật lọc cộng
tác. Ngoài ra, Chương 1 còn trình bày ngắn gọn các vấn đề còn tồn tại của hệ tư vấn lọc
cộng tác.

\textbf{Chương 2: Tư vấn dựa trên mô hình kết hợp} - Trong chương này, đồ án trình bày về mô hình kết hợp giữa hành vi đánh giá và hành
vi bình luận và cách ứng dụng mô hình kết hợp vào hệ tư vấn lọc cộng tác. Ngoài ra, nội
dung Chương 2 còn trình bày về các kỹ thuật tiền xử lý dữ liệu văn bản cùng với 3 kỹ
thuật phân loại văn bản: Naïve Bayes, Logistic Regression, SVM.

\textbf{Chương 3: Thử nghiệm và đánh giá} - Chương 3 tập trung trình bày về bộ dữ liệu được thử nghiệm, phương pháp thực
nghiệm, bộ dữ liệu được sử dụng và kết quả thực nghiệm và đánh giá.
